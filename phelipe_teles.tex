\documentclass{resume}

\usepackage[left=0.4 in,top=0.4in,right=0.4 in,bottom=0.4in]{geometry}
\usepackage[utf8]{inputenc}
\usepackage[brazil]{babel}
\usepackage{hyperref}
\usepackage{calcage}

\newcommand{\tab}[1]{\hspace{.2667\textwidth}\rlap{#1}}
\newcommand{\itab}[1]{\hspace{0em}\rlap{#1}}
\name{PHELIPE TELES DA SILVA}
\address{30 de jan. de 1996 (\calcage{1996}{01}{30}anos) \\ Rua Seabra Filho, 436, Inhoaíba, Rio de Janeiro}
\address{(21) 9 6678-8995 \\ phelipe\_teles@hotmail.com \\ \href{https://linkedin.com/in/phelipeteles}{linkedin.com/in/phelipeteles} \\ \href{phelipetls.github.io}{phelipetls.github.io} }

\begin{document}

%----------------------------------------------------------------------------------------
%   OBJECTIVE
%----------------------------------------------------------------------------------------

\begin{rSection}{OBJETIVO}
  Vaga entry-level na área de programação.

  Sou familizariado com Data Science, vindo do curso de Economia e com experiência nas
  linguagens R e Python aplicada à Econometria.

  Recentemente venho estudando Desenvolvimento Web, principalmente JavaScript,
  tanto frontend quanto backend.
  \vspace{5mm}
\end{rSection}

%----------------------------------------------------------------------------------------
%   EDUCATION SECTION
%----------------------------------------------------------------------------------------

\begin{rSection}{FORMAÇÃO}
  {\bf Ciências Econômicas} \hfill {Mar 2015 -- Dez  2020}
  \\
  Universidade Federal Rural do Rio de Janeiro.
  \\
\end{rSection}

%----------------------------------------------------------------------------------------
%   WORK EXPERIENCE SECTION
%----------------------------------------------------------------------------------------

\begin{rSection}{EXPERIÊNCIA}
  \begin{rSubsection}{Estagiário de Planejamento Comercial - Infoglobo}{Dez 2018 -- Presente}{}
  \item
  \item Criação de relatórios com PowerBI, Excel e Salesforce.
  \item Automatização de relatórios utilizando a API Analytics da
    Salesforce no VBA.
  \item Web Scraping de sites dinâmicos no VBA.
    \vspace{5mm}
  \end{rSubsection}
\end{rSection}

%----------------------------------------------------------------------------------------
%   TECHNICAL STRENGTHS SECTION
%----------------------------------------------------------------------------------------

\begin{rSection}{SOFTWARES \& LINGUAGENS DE PROGRAMAÇÃO}
  \begin{tabular}{ @{} >{\bfseries}l @{\hspace{6ex}} l }
    Básico & SQL, Bash.\\
    Intermediário & JavaScript, R, PowerBI, Salesforce, VBA.\\
    Avançado & Python, Excel.\\
  \end{tabular}
  \vspace{5mm}
\end{rSection}

%----------------------------------------------------------------------------------------
%   PROJECTS SECTION
%----------------------------------------------------------------------------------------

\begin{rSection}{PROJETOS}

  \begin{rSubsection}{\href{https://github.com/phelipetls/minesweeper.js}{Campo minado em JavaScript}}{}{}
  \item
  \item Implementado em JavaScript (backend Express) e CSS puro.
  \item Sistema de autenticação e armazenamento de estatísticas do usuário com
        PostgreSQL.
  \item Hospedado no \href{https://some-minesweeper.herokuapp.com/}{Heroku}.
    \vspace{5mm}
  \end{rSubsection}

  \begin{rSubsection}{Pacotes no Python}{}{}
  \item
  \item \href{https://github.com/phelipetls/reportforce}{\textbf{reportforce}}:
        Cliente em Python para a Analytics REST API da Salesforce com ênfase na
        extração de relatórios.
  \item \href{https://github.com/phelipetls/seriesbr}{\textbf{seriesbr}}:
        Interface para as Web APIs do BCB, IPEA e IBGE.
  \item \href{https://github.com/phelipetls/mapsbr}{\textbf{mapsbr}}:
        Auxilia na obtenção e uso de dados geoespaciais brasileiros utilizando a
        API do  IBGE e ArcGIS.
  \item Uso das bibliotecas pandas, requests e pytest.
  \vspace{5mm}
  \end{rSubsection}

\end{rSection}

%----------------------------------------------------------------------------------------
%   IDIOMS SECTION
%----------------------------------------------------------------------------------------

\begin{rSection}{IDIOMAS} \itemsep -3pt
    {Inglês avançado.}
\end{rSection}

\end{document}

% vi: nowrap
