%%%%%%%%%%%%%%%%%%%%%%%%%%%%%%%%%%%%%%%%%
% Medium Length Professional CV
% LaTeX Template
% Version 2.0 (8/5/13)
%
% This template has been downloaded from:
% http://www.LaTeXTemplates.com
%
% Original author:
% Trey Hunner (http://www.treyhunner.com/)
%
% Important note:
% This template requires the resume.cls file to be in the same directory as the
% .tex file. The resume.cls file provides the resume style used for structuring the
% document.
%
%%%%%%%%%%%%%%%%%%%%%%%%%%%%%%%%%%%%%%%%%

\documentclass{resume} % Use the custom resume.cls style

\usepackage[left=0.4 in,top=0.4in,right=0.4 in,bottom=0.4in]{geometry}
\usepackage[utf8]{inputenc}
\usepackage{hyperref}
\usepackage{calcage}

\newcommand{\tab}[1]{\hspace{.2667\textwidth}\rlap{#1}}
\newcommand{\itab}[1]{\hspace{0em}\rlap{#1}}
\name{PHELIPE TELES DA SILVA}
\address{30 de jan. de 1996 (\calcage{1996}{01}{30}anos) \\ Rua Seabra Filho, 436, Inhoaíba, Rio de Janeiro}
\address{(21) 9 6678-8995 \\ phelipe\_teles@hotmail.com \\ \href{https://linkedin.com/in/phelipeteles}{linkedin.com/in/phelipeteles} \\ \href{phelipetls.github.io}{phelipetls.github.io} }

\begin{document}

%----------------------------------------------------------------------------------------
%   OBJECTIVE
%----------------------------------------------------------------------------------------

\begin{rSection}{OBJETIVO}
    Experiências que envolam programação aplicada à análise, visualização e modelagem de dados.
    \vspace{5mm}
\end{rSection}

%----------------------------------------------------------------------------------------
%   EDUCATION SECTION
%----------------------------------------------------------------------------------------

\begin{rSection}{FORMAÇÃO}
    {\bf Ciências Econômicas} \hfill {Mar 2015 -- Jun 2020}
    \\
    Universidade Federal Rural do Rio de Janeiro.
    \\
\end{rSection}

%----------------------------------------------------------------------------------------
%   WORK EXPERIENCE SECTION
%----------------------------------------------------------------------------------------

\begin{rSection}{EXPERIÊNCIA}
    \begin{rSubsection}{Monitoria de Econometria I}{Abr 2018 -- Dez 2018}{}
    \item
    \item Contribuiu para aprofundar meu conhecimento teórico de Estatística, Economia e Matemática.
    \item Contato com softwares econométricos (EViews), para a construção e análise dos modelos.
        \vspace{5mm}
    \end{rSubsection}

    \begin{rSubsection}{Estagiário de Planejamento Comercial na Infoglobo}{Dez 2018 -- Presente}{}
    \item
    \item Criação de relatórios no Excel, PowerBI e SalesForce.
    \item Automatização de tarefas com VBA e Task Scheduler (Web Scraping, por exemplo).
        \vspace{5mm}
    \end{rSubsection}
\end{rSection}

%----------------------------------------------------------------------------------------
%   TECHNICAL STRENGTHS SECTION
%----------------------------------------------------------------------------------------

\begin{rSection}{SOFTWARES \& LINGUAGENS DE PROGRAMAÇÃO}
    \begin{tabular}{ @{} >{\bfseries}l @{\hspace{6ex}} l }
        Básico & SQL, Bash e PowerBI.\\
        Intermediário & R, Python e VBA.\\
        Avançado & Excel.\\
    \end{tabular}
    \vspace{5mm}
\end{rSection}

%----------------------------------------------------------------------------------------
%   PROJECTS SECTION
%----------------------------------------------------------------------------------------

\begin{rSection}{PROJETOS}
    \begin{rSubsection}{\href{https://github.com/phelipetls/mapsbr}{Pacote mapsbr no Python}}{Dez. 2019}{}
    \item
    \item Pacote para auxiliar na obtenção de dados geoespaciais brasileiros.
        \vspace{5mm}
    \end{rSubsection}

    \begin{rSubsection}{\href{https://github.com/phelipetls/seriesbr}{Pacote seriesbr no Python}}{Dez. 2019}{}
    \item
    \item Pacote que serve como uma interface no Python para as bases de dados do Banco Central do Brasil (BCB), Instituto de Pesquisa Econômica Aplicada (IPEA) e Instituto Brasileiro de Estatística e Geografia (IBGE).
        \vspace{5mm}
    \end{rSubsection}

    \begin{rSubsection}{\href{http://phelipetls.github.io}{Blog no Github Pages}}{Fev 2018 -- Presente}{}
    \item
    \item Blog sobre estatística, programação e matemática.
    \item Me fez aprender sobre git, HTML, CSS, Jekyll, \LaTeX, Python e R.
        \vspace{5mm}
    \end{rSubsection}
\end{rSection}

%----------------------------------------------------------------------------------------
%   IDIOMS SECTION
%----------------------------------------------------------------------------------------

\begin{rSection}{IDIOMAS} \itemsep -3pt
        {Inglês avançado.}
\end{rSection}

\end{document}

% vi: nowrap
